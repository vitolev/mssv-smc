\documentclass[a4paper, 12pt]{article}
\usepackage{lmodern}
\usepackage[T1]{fontenc}
\usepackage[utf8]{inputenc}
\usepackage{url}
\usepackage{xcolor}
\usepackage[square,numbers,sort&compress]{natbib}
\bibliographystyle{elsarticle-num}


\definecolor{munsell}{rgb}{0.0, 0.5, 0.69}
\newcommand\cmnt[1]{\textcolor{munsell}{#1}}

\topmargin=0cm
\topskip=0cm
\textheight=25cm
\headheight=0cm
\headsep=0cm
\oddsidemargin=0cm
\evensidemargin=0cm
\textwidth=16cm
\parindent=0cm
\parskip=12pt

\renewcommand{\baselinestretch}{1.2}

\begin{document}

%%%%%%%%%%%%%%%%%%%%%%%%%% Filled out by the candidate! %%%%%%%%%%%%%%%%%%%%%%%%%%
\newcommand{\ImeKandidata}{Vito} % Name
\newcommand{\PriimekKandidata}{Levstik} % Surname
\newcommand{\VpisnaStevilka}{63240453} % Enrollment number
\newcommand{\StudyProgramme}{Computer and information science, MAG} % Study programme
\newcommand{\NaslovBivalisca}{Ulica Jožefe Lackove 28a, 2250 Ptuj, Slovenija} % the candidate address
\newcommand{\SLONaslov}{Ocenjevanje volatilnosti z Markovskim prehajanjem na finančnih trgih s sekvenčnimi Monte Carlo metodami} % Slovenian title
\newcommand{\ENGNaslov}{Estimating Markov switching volatility in financial markets using sequential Monte Carlo methods} % English title
%%%%%%%%%%%%%%%%%%%%%%%%%% End of filling in  %%%%%%%%%%%%%%%%%%%%%%%%%%


\newcommand{\ProblemParagraph}{\noindent\textbf{Problem \& State of the Art.} }
\newcommand{\ContributionsParagraph}{\noindent\textbf{Expected Contributions / Technical outcome. } }
\newcommand{\MethodologyParagraph}{\noindent\textbf{Methodology \& Validation. } }

\newcommand{\Kandidat}{\ImeKandidata~\PriimekKandidata}
\noindent
\Kandidat\\
\NaslovBivalisca \\
Study programme: \StudyProgramme \\
Enrollment number: \VpisnaStevilka
\bigskip

{\bf Committee for Student Affairs}\\
Univerza v Ljubljani, Fakulteta za računalništvo in informatiko\\
Večna pot 113, 1000 Ljubljana\\

{\Large\bf
{\centering
    The master’s thesis topic proposal \\%[2mm]
\large Candidate: \Kandidat \\[10mm]}}


I, \Kandidat, a student of the 2nd cycle study programme at the Faculty of computer and information science, am submitting a thesis topic proposal to be considered by the Committee for Student Affairs with the following title:

%\hfill\begin{minipage}{\dimexpr\textwidth-2cm}
Slovenian: {\bf \SLONaslov}\\
English: {\bf \ENGNaslov}
%\end{minipage}

This topic was already approved last year: \textit{NO}
 
I declare that the mentors listed below have approved the submission of the thesis topic proposal described in the remainder of this document.

I would like to write the thesis in English with the following reason: It is encouraged to write the thesis in English for Data Science students, hence I would like to follow this recommendation.

I propose the following mentor:

%%%%%%%%%%%%%%%%%%%%%%%%%% Filled in by the candidate! %%%%%%%%%%%%%%%%%%%%%%%%%%
\hfill\begin{minipage}{\dimexpr\textwidth-2cm}
Name, surname and title: izr. prof. dr. Jure Demšar\\
Institution: Fakulteta za računalništvo in informatiko, Univerza v Ljubljani
E-mail: jure.demsar@fri.uni-lj.si
\end{minipage}

%%%%%%%%%%%%%%%%%%%%%%%%%% End of filling in %%%%%%%%%%%%%%%%%%%%%%%%%%

\bigskip

\hfill Ljubljana, \today. \newpage


\section*{Key-words}

Bayesian inference, sequential Monte Carlo, Markov switching models, financial time series, stochastic volatility

\section*{Detailed thesis proposal}

\ProblemParagraph
Understanding and forecasting financial market volatility is essential for risk management, portfolio allocation, and derivatives pricing \cite{AndersenBollerslev1998}. 
Market volatility often exhibits sudden changes, alternating between low and high volatility regimes. 
Markov switching stochastic volatility (MSSV) models capture these dynamics by modeling latent volatility and discrete market regimes \cite{SoLamLi1998}. 
In such models, sequential Monte Carlo (SMC) methods are tools for estimating latent states \cite{doucet2001sequential}, but many approaches rely on known parameters or simplified dynamics \cite{VirbickaitėLopes2019,CarvalhoLopes2007,BaoChiarellaKang2018}. 
Full Bayesian inference that jointly estimates latent states and all model parameters is less commonly implemented, particularly for real-world financial time series, leaving scope for methodological development and empirical evaluation.

\ContributionsParagraph
This thesis will develop a Bayesian inference framework for MSSV models using SMC methods. The main contributions are:
(1) the implementation of Bayesian inference framework using PMCMC \cite{AndrieuDoucetHolenstein2010} and SMC² \cite{ChopinJacobPapaspiliopoulos2012} to jointly infer latent volatility, regime states, and model parameters;
(2) a comparison of SMC-based inference with conventional MCMC methods; and
(3) an analysis on equity and cryptocurrency data evaluating the forecasting performance of MSSV models relative to standard volatility benchmarks such as GARCH \cite{Bollerslev1986}.

\MethodologyParagraph
The methodology will be implemented in Python and validated using both simulated and real financial time series. Performance of SMC methods will be evaluated against conventional MCMC methods, with a focus on computational cost and estimation efficiency.
In addition, the MSSV model’s forecasting ability will be compared to standard volatility models such as GARCH, using realized volatility as the benchmark.

 
\bibliography{bib/references}
 
\end{document}
